\documentclass{ExpressiveResume}


% ----- Resume -----
\begin{document}

% ----- Name + Contact Information -----
\resumeheader[
    firstname=Qifa(Richard),
    lastname=Wang,
    email=qifaw2000@gmail.com,
    phone=617-816-5834,
    linkedin=qifa-wang-28a180170,
    github=rthelionheart24,
    fixobjectivespacing=true
]

\objective{Motivated engineering graduate passionate about computer architecture, SW/HW co-design, accelerated computing, HPC, quantum computing, and software engineering. Quick to adapt to new technologies and concepts.}
% ----- Education -----
\section{Education}

\experience{Master of Science in Engineering}{Computer Science \& Engineering}{Aug 2023}{May 2025}{
    \noindent Rackham Graduate School, University of Michigan \hfill GPA: 4.00
}

\experience{Bachelor of Science in Engineering}{Computer Science, minors in Math and Physics}{Aug 2020}{May 2023}{
    \noindent College of Engineering, University of Michigan \hfill \textit{Summa Cum Laude, Dean's List, University Honors}, GPA: 3.77
}


\section{Skills}
\noindent\achievement{\textbf{Coursework:} Computer Architecture and Microarchitecture,
    Parallel Computing and Architecture, GPU Programming, Machine
    Learning, Quantum Computing and Architecture, Compiler Design, Data Structure and Algorithms, Operating System, Web Systems}
\achievement{\textbf{Languages:} C/C++, Python, Go, Rust, Java, CUDA, OpenMP, MPI,
    Verilog/SystemVerilog, Chisel, Tcl, Javascript, TypeScript}
\achievement{\textbf{Frameworks and tools:} PyTorch, CUDA-Q, Qiskit, AWS, Docker, Chipyard, GPGPU-Sim, Synopsys (DC, VCS, Verdi), GNU, LLVM, Valgrind}
\achievement{\textbf{Multilingual:} English, Mandarin, Cantonese, French}



% ----- Work Experience -----
\section{Work Experience}

\experience{Hardware Technology Intern}{Apple Inc.}{May 2024}{Aug 2024}{
    \achievement{
        Implemented \tech{PPROC method} to evaluate hardware coverage on
        pixel processing module down to bit-field-level. Coverage result
        provides feedback and guidance for arch test improvement.
        Constructed feedback loop using PPROC
        to optimize test coverage by adjusting parameters.
        Internalize hardware coverage PPROC method into \tech{C/C++} testing
        infrastructure to run and provide feedback on-the-fly.
    }
}

\experience{Graduate Student Instrctor}{University of Michigan}{May 2023}{Present}{
    \achievement{
        EECS 498: Quantum Computing - Discussions and labs on quantum theories, algorithms and circuits using \tech{Qiskit}.
    }
    \achievement{
        CSE 587: Parallel Computing - Discussions and labs on \tech{OpenMPI, OpenMP, and CUDA}.
    }
}

\experience{Software \& Algorithm Development Intern}{Werfen}{May 2021}{Aug 2021}{

    \achievement{
        Drafted roadmaps and formalized criteria and limitations for Mercury Algorithm Prototype (MAP). Engineered \tech{embedded} modules, \tech{front-end web tool}, and GUI for the prototype using \tech{C/C++ and Qt framework}.
    }
}


% ----- Technical Projects -----
\section{Technical Projects}

\experience{R10K Out-of-Order Processor based on RISC-V ISA}{Architecture}{Jan 2024}{May 2024}{

    \achievement{
        Designed and implemented out-of-order superscalar
        processor with N-way execution at \tech{RTL level} using \tech{SystemVerilog}, featuring \tech{Tomasulo's algorithm},
        instruction and data caches (prefetching, associative, and non-blocking with victim cache),
        G-share best branch predictor, return address stack, and reservation
        stations. Optimized performance with early tag broadcast and a
        robust memory hierarchy including a load-store queue with data
        forwarding. Achieved a 30\% improvement in CPI and a clock period of 15.5ns through various
        architectural enhancements and pipeline optimizations. Verified
        and tested the design using \tech{Synopsys DC} and exhaustive
        benchmarks.
    }
}
\experience{Batched Quantum Circuit Simulation on CUDA-ready GPU}{GPU}{Sep 2023}{May 2024}{
    \achievement{
        Developed a \tech{CUDA-based} quantum simulation framework for
        performing batched quantum experiments on GPUs, enabling
        parallel execution of quantum circuits with varying gate counts
        and types. Implemented a \tech{synchronization strategy} to handle
        non-deterministic quantum operators and efficient
        \tech{shot-branching}. Achieved \tech{super-linear speedup} in runtime
        performance and optimized memory usage through careful task
        batching and state management.
    }

}

\experience{Compiler Optimization for CUDA Memory Coalescing
    (COALDA)}{Compiler}{Sep 2023}{Dec 2023}{
    \achievement{
        Developed an \tech{IR-level} CUDA compiler optimization tool using \tech{AST and canonical forms} to transform uncoalesced memory accesses into coalesced patterns,
        achieving 9x L2 cache writeback reduction and 6.4x read bandwidth improvement, validated through \tech{Clang-NVCC} integration and \tech{NVIDIA Nsight Compute}.
    }
}


\experience{Linux-Based Operating System}{Operating System}{Jan 2023}{May 2023}{
    \achievement{
        Developed a \tech{custom thread library} to simulate multi-cpu, multi-threaded
        execution using \textbf{C/C++}. Implemented a sophisticated \tech{kernel pager
        system} for efficient management of applications' virtual memory,
        encompassing the creation, copying, destruction, and allocation of address spaces.
        Engineered robust, \tech{multi-threaded network file server} for reliable
        data exchange. Designed a hierarchical file system with comprehensive
        access control and fine-grained locking mechanisms to secure file
        ownership and permissions.
    }
}

\end{document}