\documentclass{ExpressiveResume}


% ----- Resume -----
\begin{document}

% ----- Name + Contact Information -----
\resumeheader[
    firstname=Qifa(Richard),
    lastname=Wang,
    email=qifaw2000@gmail.com,
    phone=617-816-5834,
    linkedin=qifa-wang-28a180170,
    github=rthelionheart24,
    fixobjectivespacing=true
]

\objective{Highly motivated engineering graduate with strong foundation in computer science and engineering. Passionate about computer architecture, SW/HW co-design, accelerated computing, HPC, quantum computing, and software engineering. Quick to adapt to new technologies and concepts.}

% ----- Education -----
\section{Education}

\experience{Master of Science in Engineering}{Computer Science \& Engineering}{Aug 2023}{May 2025}{
    \noindent Rackham Graduate School, University of Michigan \hfill GPA: 4.00
    \achievement{Graduate Student Instructor for EECS 498: Quantum Computing and EECS 587: Parallel Computing}
}

\experience{Bachelor of Science in Engineering}{Computer Science, minors in Math and Physics}{Aug 2020}{May 2023}{
    \noindent College of Engineering, University of Michigan \hfill \textit{Summa Cum Laude, Dean's List, University Honors}, GPA: 3.77
}

\section{Skills}
\noindent\achievement{\textbf{Coursework:} Computer Architecture and Microarchitecture,
    Parallel Computing and Architecture, GPU Programming,
    Compiler Design, Data Structure and Algorithms, Operating System, Machine
    Learning, Web Systems, Quantum Computing and Architecture}
\achievement{\textbf{Languages:} C/C++, Python, Go, Rust, Java, CUDA, OpenMP, MPI,
    Verilog/SystemVerilog, Chisel, Tcl, Javascript, TypeScript}
\achievement{\textbf{Frameworks and tools:} PyTorch, CUDA-Q, Qiskit, AWS, Docker, Chipyard, GPGPU-Sim, Synopsys (DC, VCS, Verdi), GNU, LLVM, Valgrind}



% ----- Work Experience -----
\section{Work Experience}

\experience{Hardware Technology Intern}{Apple Inc.}{May 2024}{Aug 2024}{
    \achievement{
        Implemented PPROC method to evaluate hardware coverage on
        pixel processing module down to bit-field-level. Coverage result
        provides feedback and guidance for arch test improvement.
        Constructed feedback loop using PPROC
        to optimize test coverage by adjusting parameters.
        Internalize hardware coverage PPROC method into C/C++ testing
        infrastructure to run and provide feedback on-the-fly.
    }
}

\experience{Software \& Algorithm Development Intern}{Werfen}{May 2021}{Aug 2021}{

    \achievement{
        Drafted roadmaps and formalized criteria and limitations for Mercury Algorithm Prototype (MAP). Engineered embedded modules, front-end web tool, and GUI for the prototype using C/C++ and Qt framework.
    }
}


% ----- Technical Projects -----
\section{Technical Projects}

\experience{R10K Out-of-Order Processor Redesign with RISC-V ISA}{Team
    Design}{Jan 2024}{May 2024}{

    \achievement{Engineered advanced features like N-way superscalar, GShare-Best branch
        predictor, early tag broadcasting, and return address stack to reduce
        latency. Revamped memory hierarchy using
        prefetching, associative, multi-ported and non-blocking
        cache, along with victim caching strategies. Optimized dependent memory
        operations with a data-forwarding load-store queue.
    }
    \achievement{
        Designed, implemented and verified microarchitecture at RTL level
        for reservation stations, Icache, Dcache, and load-store
        queue. Successfully simulated and synthesized the processor and
        passed 100\% of test suites. Performanced ranked top 25\% among all.

    }
}

\experience{Batched Quantum Circuit Simulation on CUDA-ready GPU}{Personal
    Project}{Sep 2023}{May 2024}{
    \achievement{
        Engineered a CUDA framework for efficient, batched simulation
        of diverse quantum circuits on GPU in parallel.
    }
    \achievement{
        Achieved super-linear enhancements in both simulation efficiency and scalability,
        identifying potential bottleneck areas and proposing corresponding
        optimization strategies.
    }
}

\experience{Compiler Optimization for CUDA Memory Coalescing
    (COALDA)}{Team Design}{Sep 2023}{Dec 2023}{
    \achievement{
        Boosted CUDA program efficiency by optimizing memory access
        patterns, targeting and restructuring uncoalesced
        memory accesses. Thorough \textbf{static analysis}
        and performance evaluations showed a notable reduction in L2 cache
        bandwidth usage.
    }
    \achievement{
        Established an NVCC-Clang compilation pipeline to make CUDA kernel
        available for IR-level optimization.
    }
}


\experience{Linux-Based Operating System}{Team Design}{Jan 2023}{May 2023}{
    \achievement{
        Developed a custom thread library to simulate multi-cpu, multi-threaded
        execution using \textbf{C/C++}. Implemented a sophisticated kernel pager
        system for efficient management of applications' virtual memory,
        encompassing the
        creation, copying, destruction, and allocation of address spaces.
        Engineered robust, multi-threaded network file server for reliable
        data
        exchange. Designed a hierarchical file system with comprehensive
        access control and fine-grained locking mechanisms to secure file
        ownership and permissions.
    }
}

\section{Personal}
\achievement{\textbf{Authorized to work for any employer in the US}}
\achievement{Competitive saber fencing athlete who also loves alpine skiing and playing tennis}
\achievement{Founder of BeaverWorks engineering club with MIT Lincoln Lab and BAE \& System as sponsors}
\achievement{Multilingual: English, Mandarin, Cantonese, French}


\end{document}