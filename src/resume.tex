\documentclass{ExpressiveResume}


% ----- Resume -----
\begin{document}

% ----- Name + Contact Information -----
\resumeheader[
    firstname=Qifa(Richard),
    lastname=Wang,
    email=qifaw2000@gmail.com,
    phone=617-816-5834,
    linkedin=qifa-wang-28a180170,
    github=rthelionheart24,
    fixobjectivespacing=true
]

\objective{Motivated engineering graduate passionate about computer architecture, SW/HW co-design, accelerated computing, HPC, quantum computing, and software engineering. Quick to adapt to new technologies and concepts.}
% ----- Education -----
\section{Education}

\experience{Master of Science in Engineering}{Computer Science \& Engineering}{Aug 2023}{May 2025}{
    \noindent Rackham Graduate School, University of Michigan \hfill GPA: 4.00
    \achievement{Graduate Student Instructor for Quantum Computing and Parallel Computing}
}

\experience{Bachelor of Science in Engineering}{Computer Science, minors in Math and Physics}{Aug 2020}{May 2023}{
    \noindent College of Engineering, University of Michigan \hfill \textit{Summa Cum Laude, Dean's List, University Honors}, GPA: 3.77
}


\section{Skills}
\noindent\achievement{\textbf{Coursework:} Computer Architecture and Microarchitecture,
    Parallel Computing and Architecture, GPU Programming, Machine
    Learning, Quantum Computing and Architecture, Compiler Design, Data Structure and Algorithms, Operating System, Web Systems}
\achievement{\textbf{Languages:} C/C++, Python, Go, Rust, Java, CUDA, OpenMP, MPI,
    Verilog/SystemVerilog, Chisel, Tcl, Javascript, TypeScript}
\achievement{\textbf{Frameworks and tools:} PyTorch, CUDA-Q, Qiskit, AWS, Docker, Chipyard, Synopsys (DC, VCS, Verdi), GNU, LLVM, Valgrind}
\achievement{\textbf{Multilingual:} English, Mandarin, Cantonese, French}



% ----- Work Experience -----
\section{Work Experience}

\experience{Hardware Technology Intern}{Apple Inc.}{May 2024}{Aug 2024}{
    \achievement{
        Implemented PPROC method to evaluate hardware coverage on
        pixel processing module down to bit-field-level. Output indicated 74\% coverage and
        identifies 22\% critical under-covered areas; arch test suites optimized upon communication with architecture team and full coverage achieved. 
        Output exposed critical bugs in architectural C-model that was later resolved by maintainers.
    }
        
    \achievement{
        Constructed heuristic-based algorithm using PPROC metadata
        to optimize arch test coverage using dynamic parameter adjustment.
        Integrate hardware coverage PPROC method into \tech{C/C++} bare-metal testing
        infrastructure and architectural C-model to run and provide feedback in real time. 
    }
}

\experience{Software \& Algorithm Development Intern}{Werfen}{May 2021}{Aug 2021}{

    \achievement{
        Improved Mercury, the point-of-care diagnostic tool for blood coagulation.
        Engineered core embedded modules to process customized medical algorithm and monitor blood data; engineered GUI module for real-time feedback and data visualization using \tech{C/C++ and Qt framework}. Built from scratch a \tech{real-time data processing pipeline} to handle, process data from medical devices and display results on GUI; prototype served as starting point for the next-gen product.
    }

}


% ----- Technical Projects -----
\section{Technical Projects}

\experience{R10K Out-of-Order Processor based on RISC-V ISA}{Architecture}{Jan 2024}{May 2024}{

    \achievement{
        Designed and implemented out-of-order superscalar
        processor with N-way execution at \tech{RTL level} using \tech{SystemVerilog}, featuring \tech{Tomasulo's algorithm},
        instruction and data caches (prefetching, associative, and non-blocking with victim cache),
        G-share best branch predictor, return address stack, and reservation stations. 
    }
    \achievement{
        Optimized performance with early tag broadcast and robust memory hierarchy including load-store queue with data forwarding. 
        Achieved 30\% improvement in CPI and clock period of 15.5ns. Verified and tested the design using \tech{Synopsys DC} and exhaustive benchmarks including 
        alexnet, dft, matrix multiplication, and saxpy.
    }
}
\experience{Batched Quantum Circuit Simulation on CUDA-ready GPU}{GPU}{Sep 2023}{May 2024}{
    \achievement{
        Developed a \tech{CUDA-based} quantum simulation framework for
        performing batched quantum experiments on GPUs, enabling
        parallel execution of quantum circuits with varying gate counts
        and types. Implemented \tech{synchronization strategy} to handle
        non-deterministic quantum operators and efficient \tech{shot-branching}. 
    }
    \achievement{
        Achieved \tech{superlinear speedup} in runtime
        performance and optimized memory usage through task
        batching and state management.
    }
}
\experience{Compiler Optimization for CUDA Memory Coalescing
    (COALDA)}{Compiler}{Sep 2023}{Dec 2023}{
    \achievement{
        Developed a IR-level CUDA compiler optimization tool using \tech{AST and canonical forms} to transform uncoalesced memory accesses into coalesced patterns.
        Implemented NVCC-LLVM pipeline that allows for seamless integration with existing CUDA toolchains and optimziation to LLVM IR. 
        
    }
    \achievement{
        Achieved 9x L2 cache writeback reduction and 6.4x read bandwidth improvement, validated through \tech{NVIDIA Nsight Compute}.
    }
}


\experience{Linux-Based Operating System}{Operating System}{Jan 2023}{May 2023}{
    \achievement{
        Developed a \tech{custom thread library} to support multi-cpu, multi-threaded
        execution using \textbf{C/C++}.
    } 
    \achievement{
        Implemented a \tech{kernel pager} 
        for efficient management of applications' virtual memory using both Swap and physical memory.
    }
    \achievement{
        Engineered a \tech{multi-threaded network file server} for reliable
        data exchange. Designed a hierarchical file system with secure file
        ownership and permissions and fine-grained locking mechanisms.
    }
}

\end{document}